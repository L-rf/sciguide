\documentclass[]{book}
\usepackage{lmodern}
\usepackage{amssymb,amsmath}
\usepackage{ifxetex,ifluatex}
\usepackage{fixltx2e} % provides \textsubscript
\ifnum 0\ifxetex 1\fi\ifluatex 1\fi=0 % if pdftex
  \usepackage[T1]{fontenc}
  \usepackage[utf8]{inputenc}
\else % if luatex or xelatex
  \ifxetex
    \usepackage{mathspec}
  \else
    \usepackage{fontspec}
  \fi
  \defaultfontfeatures{Ligatures=TeX,Scale=MatchLowercase}
\fi
% use upquote if available, for straight quotes in verbatim environments
\IfFileExists{upquote.sty}{\usepackage{upquote}}{}
% use microtype if available
\IfFileExists{microtype.sty}{%
\usepackage{microtype}
\UseMicrotypeSet[protrusion]{basicmath} % disable protrusion for tt fonts
}{}
\usepackage[margin=1in]{geometry}
\usepackage{hyperref}
\hypersetup{unicode=true,
            pdftitle={现代科研指北},
            pdfauthor={于淼},
            pdfborder={0 0 0},
            breaklinks=true}
\urlstyle{same}  % don't use monospace font for urls
\usepackage{natbib}
\bibliographystyle{apalike}
\usepackage{longtable,booktabs}
\usepackage{graphicx,grffile}
\makeatletter
\def\maxwidth{\ifdim\Gin@nat@width>\linewidth\linewidth\else\Gin@nat@width\fi}
\def\maxheight{\ifdim\Gin@nat@height>\textheight\textheight\else\Gin@nat@height\fi}
\makeatother
% Scale images if necessary, so that they will not overflow the page
% margins by default, and it is still possible to overwrite the defaults
% using explicit options in \includegraphics[width, height, ...]{}
\setkeys{Gin}{width=\maxwidth,height=\maxheight,keepaspectratio}
\IfFileExists{parskip.sty}{%
\usepackage{parskip}
}{% else
\setlength{\parindent}{0pt}
\setlength{\parskip}{6pt plus 2pt minus 1pt}
}
\setlength{\emergencystretch}{3em}  % prevent overfull lines
\providecommand{\tightlist}{%
  \setlength{\itemsep}{0pt}\setlength{\parskip}{0pt}}
\setcounter{secnumdepth}{5}
% Redefines (sub)paragraphs to behave more like sections
\ifx\paragraph\undefined\else
\let\oldparagraph\paragraph
\renewcommand{\paragraph}[1]{\oldparagraph{#1}\mbox{}}
\fi
\ifx\subparagraph\undefined\else
\let\oldsubparagraph\subparagraph
\renewcommand{\subparagraph}[1]{\oldsubparagraph{#1}\mbox{}}
\fi

%%% Use protect on footnotes to avoid problems with footnotes in titles
\let\rmarkdownfootnote\footnote%
\def\footnote{\protect\rmarkdownfootnote}

%%% Change title format to be more compact
\usepackage{titling}

% Create subtitle command for use in maketitle
\newcommand{\subtitle}[1]{
  \posttitle{
    \begin{center}\large#1\end{center}
    }
}

\setlength{\droptitle}{-2em}
  \title{现代科研指北}
  \pretitle{\vspace{\droptitle}\centering\huge}
  \posttitle{\par}
  \author{于淼}
  \preauthor{\centering\large\emph}
  \postauthor{\par}
  \predate{\centering\large\emph}
  \postdate{\par}
  \date{2018-01-27}

\usepackage{booktabs}
\usepackage{ctex}
\setCJKmainfont{FangSong}
\setCJKmonofont{KaiTi}
\setCJKsansfont{SimHei}

\begin{document}
\maketitle

{
\setcounter{tocdepth}{1}
\tableofcontents
}
\chapter{前言}

才疏学浅,不知何为真,仅通少错之法,故不敢言南,仅指北。或曰:现代科研挖坑/跳坑指南

\chapter{科研在搞什么鬼}\label{intro}

\section{科研老鸭汤-科学哲学沿革}\label{-}

\subsection{古希腊}

\subsection{中世纪}

\subsection{1500年以后}

\subsection{逻辑实证主义}

\subsection{否证主义}

\subsection{历史主义}

\subsection{无政府主义}

\subsection{实用主义}

\subsection{其他}

\section{科研职业化-问题为导向}\label{-}

\section{科研精细化-体面的博士}\label{-}

\section{科学知识的五个层次}

\subsection{背景组}

\subsection{已知的已知组}

\subsection{已知的未知组}

\subsection{未知的已知组}

\subsection{未知的未知组}

\section{知识体系的时间构建}

\chapter{科研现状概览}\label{view}

\section{国内版}

\section{国际版}

\section{趋势}

\subsection{科学方法}

\subsection{数据驱动的科研}

\subsection{社交网络中的科研}

\chapter{思维工具篇}\label{thought}

\section{科学思维}

\subsection{规律的失效}

\subsection{哈森奇效应}

\subsection{观察研究的敌人-反馈}\label{-}

\section{模型思维}

\subsection{可编程}

\subsection{抽象}

\subsection{交互作用}

\section{统计思维}

\section{估算法}

\subsection{费米估计}

\chapter{实验}

\section{实验设计原则}

\section{定性实验}

\section{定量实验}

\section{思想实验}

\chapter{数据处理}

\section{多重比较}

\section{多重检验}

\section{回归}

\section{预测}

\section{仿真}

\section{可视化}

\chapter{文献}

\section{文献管理}

\subsection{从无到有}

\subsection{从有到精}

\subsection{从精到用}

\section{信息收集}

\subsection{Zotero}\label{zotero}

\subsection{Mendeley}\label{mendeley}

\subsection{EndNote}\label{endnote}

\section{文本挖掘}

\subsection{关键词}

\subsection{作者}

\subsection{时空分布}

\subsection{影响力}

\section{荟萃分析}

\chapter{学术生活}

\section{学术出版}

\subsection{期刊论文}

\subsection{会议摘要}

\subsection{专著}

\subsection{专利}

\subsection{软件}

\section{学术会议}

\subsection{口头报告}

\subsection{海报报告}

\subsection{听报告}

\section{审稿}

\section{学术合作}

\subsection{数据共享}

\subsection{社交网络}

\section{讲课}

\section{课题组管理}

\subsection{科研的创业隐喻}

\section{学术声誉}

\section{学术道德/伦理}

\section{案例}

\chapter{Placeholder}\label{placeholder}

\chapter{专题一:基于互联网数据的科研}

\chapter{专题二:现代科研兵刃谱}

\section{文本编辑}

\section{学术交流}

\section{审稿}\label{-1}

\section{数据分享}

\section{代码管理}

\subsection{R包管理}\label{r}

\chapter{专题三:拖延症}

\section{敌人只有一个-与未来的自己博弈}\label{-}

\section{专注与转移}

\section{鸵鸟战术}

\section{香肠战术}

\section{紧急-重要四象限}\label{-}

\bibliography{book.bib,packages.bib}


\end{document}
